\documentclass[12pt, oneside]{article} 
\usepackage{amsmath, amsthm, amssymb, calrsfs, wasysym, verbatim, bbm, color, graphicx, geometry, fancyhdr, url}
\usepackage[italian]{babel}

\geometry{tmargin=.75in, bmargin=.75in, lmargin=.75in, rmargin = .75in}


\author{RAMtastic6}

%Intestazione
\pagestyle{fancy}
\fancyhf{}
\fancyhead[R]{Gruppo 14 RAMtastic6\\ramtastic6@gmail.com}
\fancyfoot[C]{\thepage}

% Linea intestazione
\renewcommand{\headrulewidth}{0pt} 

% Intestazione documento
\begin{document}
% Salta la prima pagina per l'intestazione
\thispagestyle{empty}
\title{Norme Di Progetto}
\maketitle
\begin{figure}[h]
  \centering
  \includegraphics[scale=0.3]{logo.png}
\end{figure}
\begin{center}
    email: ramtastic6@gmail.com
\end{center}

% Informazioni sul documento
\section*{Informazioni sul documento}
\begin{tabular}{ll}
Versione: & 0.2.1 \\
Redattori: &  Visentin S.  Basso L. \\
Verificatori: & Davide B. Michele Z. Leonardo B. Riccardo Z. \\
Destinatari: & T. Vardanega, R. Cardin, Imola Informatica \\
Uso: & Interno
\end{tabular}
\newpage

% Registro dei cambiamenti
\section*{Registro dei Cambiamenti - Changelog}
\begin{tabular}{|c|c|c|c|p{6cm}|}
\hline
\textbf{Versione} & \textbf{Data} & \textbf{Autore} & \textbf{Verificatore} & \textbf{Dettaglio} \\
\hline
v.0.2.1 & 2023-11-12 & Leonardo B. & da verificare & Stesura della sottosezione 3.2 della sezione relativa alla documentazione \\
\hline
v.0.2.0 & 2023-11-12 & Leonardo B. & Riccardo Z. & Stesura della sezione 3.1 (Documentazione) e delle sottosezioni relative ad essa \\
\hline
v.0.1.0 & 2023-10-30 & Samuele V. & Filippo T. & Prima versione \\
\hline
\end{tabular}
\newpage

% Sommario
\tableofcontents
\newpage
\section{Introduzione}
\subsection{Riferimenti}
\subsubsection{Riferimenti normativi}
\begin{enumerate}
    \item Capitolato d'appalto C3: \\ \url{https://www.math.unipd.it/~tullio/IS-1/2023/Progetto/C3.pdf}
\end{enumerate}
\newpage

\section{Processi primari}
\newpage

\section{Processi di supporto}
\subsection{Documentazione}
\subsubsection{Obiettivi}
\subsubsection{Tipologie di documenti}
I documenti prodotti possono essere classificati in due classi principali: ad uso interno e ad uso esterno; la prima categoria comprende
\begin{itemize}
    \item Verbali interni (i quali non necessitano di versionamento)
    \item Norme di progetto
\end{itemize}
La seconda categoria di documenti comprende:
\begin{itemize}
    \item Verbali esterni
    \item Piano di qualifica
    \item Piano di progetto
    \item Analisi dei requisiti
\end{itemize}
E' importante sottolineare che tutti i documenti sopracitati sono ufficiali e devono essere, quindi, preventivamente approvati da verificatori designati.
\subsubsection{Ciclo di vita di un documento}
Un documento segue le seguenti fasi di produzione:
\begin{itemize}
    \item Stesura: uno o più redattori si occupano di redigere il contenuto del documento.
    \item Verifica: uno o più membri del gruppo diversi da quelli che hanno redatto il documento lo verificano dopo le modifiche apportate.
    \item Approvazione: durante questa fase, il responsabile di progetto può decidere se approvare l'inclusione di un particolare documento all'interno del repository. Nel caso in cui il documento non venga approvato, si ritorna alla fase di stesura.
\end{itemize}
\subsubsection{Template}
Il gruppo ha scelto di utilizzare template LaTeX per la produzione della documentazione. Per visualizzare la struttura e utilizzare i template, è sufficiente accedere alla cartella \textit{documentazione\_interna} all'interno del repository Github.

\subsubsection{Struttura di un documento}
Un documento all'interno del nostro contesto segue una struttura ben definita, le sue sezioni principali includono:
\begin{itemize}
    \item Prima pagina: contiene il nome del gruppo e informazioni in merito al documento: uso, destinatari, redattori, verificatori, versione
    \item Indice: elenco strutturato dei contenuti del documento
    \item Registro dei cambiamenti: una tabella contente informazioni di versionamento relative al documento attuale; queste includono: la versione, la data, l'autore, il verificatore e una breve descrizione in merito alle modifiche apportate al documento.
    \item Intestazione: all'interno di essa vi sono il nome e l'indirizzo email del gruppo.
\end{itemize}

N.B: I verbali non contengono il registro dei cambiamenti.

\subsubsection{Strumenti}
Per la creazione e la gestione della struttura dei documenti è stato deciso di utilizzare Overleaf, un editor LaTeX online che permette la stesura collaborativa dei documenti.

\subsection{Controllo di configurazione}
\subsubsection{versionamento}
capire come gestire i numeri di versione.
\subsubsection{Git e Github}
Il gruppo RAMtastic6 ha scelto di utilizzare come strumento di versionamento \emph{GitHub} e di utilizzare \emph{Git} come strumento per collegarsi alla repository GitHub.
Inoltre si è scelto di utilizzare gitflow come flusso di lavoro il quale verrà discusso in modo dettagliato in seguito.
Link per il download dell'installer di Git: \url{https://git-scm.com/downloads}.\\
Inoltre, a questo link: \url{https://rogerdudler.github.io/git-guide/index.it.html} si troverà una breve guida su come utilizzare git.
In sintesi si elencano i pricipali comandi:
\begin{itemize}
    \item git clone \emph{link repo}\\
    questo comando copierà la repository di github in locale
    \item git add \emph{nome file} (oppure "." per includere tutti i file)\\
    \emph{git add} aggiunge le modifice apportate ai files del repository, senza eseguire questo comando un file aggiunto, eliminato o modificato non verrà salvato nella repository remota tramite il comando \emph{git push}.
    \item git commit -m "messaggio" \\
    salva le modifche apportate ai files in locale associando a quello stato un messaggio
    \item git push origin \emph{origine} \\
    salva le modifiche in remoto nel branch specificato
    \item git pull \\
    permette di aggiornare la repo in locale e in caso di necessità esegue il merge
\end{itemize}
\subsubsection{Struttura del repository}
La strutta della repository per i documenti deve essere:
\begin{itemize}
    \item documenti
    \begin{itemize}
        \item CANDIDATURA
        \item RTB
        \item PB
    \end{itemize}
    \item diari\_di\_bordo
    \item documenti\_interni
\end{itemize}
\subsubsection{Controllo di flusso}
Installando git si avrà accesso anche ai comandi di git flow. \\Link tutorial di gitflow: \\ \url{https://www.atlassian.com/git/tutorials/comparing-workflows/gitflow-workflow}\\ Link su come gestire le feature:\\ \url{http://danielkummer.github.io/git-flow-cheatsheet/}.

\newpage
\section{Processi organizzativi}
Viene adottato lo snake case per nominare le cartelle all'interno del repository.

\end{document}

