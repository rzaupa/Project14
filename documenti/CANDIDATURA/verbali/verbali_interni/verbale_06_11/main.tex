\documentclass[12pt, oneside]{article} 
\usepackage{amsmath, amsthm, amssymb, calrsfs, wasysym, verbatim, bbm, color, graphicx, geometry, fancyhdr}
\usepackage[italian]{babel}

\geometry{tmargin=.75in, bmargin=.75in, lmargin=.75in, rmargin = .75in}


\author{RAMtastic6}

%Intestazione
\pagestyle{fancy}
\fancyhf{}
\fancyhead[R]{Gruppo 14 RAMtastic6\\ramtastic6@gmail.com}
\fancyfoot[C]{\thepage}

% Intestazione documento
\begin{document}
% Salta la prima pagina per l'intestazione
\thispagestyle{empty}
\title{Riunione di team}
\date{06 novembre 2023} % cambiare la data del verbale in questo slot

\maketitle
\begin{figure}[h]
	\centering
	\includegraphics[scale=0.3]{logo.png}
	\label{}
\end{figure}
\begin{center}
    email: ramtastic6@gmail.com
\end{center}

% Informazioni sul documento
\section*{Informazioni sul documento}
\begin{tabular}{ll}
Versione: & 1.1 \\
Redattori: & Samuele V. Filippo T. \\
Verificatori: & 
\end{tabular}
\newpage

% Sommario
\tableofcontents
\newpage

% Contenuto
\section{Partecipanti}

\begin{center}
	\begin{tabular}{ | c | c | c | }
		\hline
		NOME & DURATA \\ 
		\hline
		Leonardo Basso & 2h \\  
		\hline 
		Davide Brotto & 2h \\   
        \hline
        Riccardo Zaupa & 2h \\   
		\hline
        Michele Zambon & 2h \\   
        \hline
        Samuele Visentin & 2h \\
        \hline
	\end{tabular}
\end{center}

\section{Riassunto dell'incontro}
\subsection{Ristrutturazione del repository}
Durante l’incontro è stata discussa la necessità di modificare la struttura della repo in modo tale da renderla più accessibile sia dai membri del gruppo che dal docente; le principali modifiche apportate sono le seguenti:
\begin{itemize}
    \item[1)] Modifica del file README.md, contenente i link ai seguenti documenti:
    \begin{itemize}
        \item[a)] verbali esterni e interni
        \item[b)] lettera di presentazione
        \item[c)] preventivo costi e impegni assunti
        \item[d)] documento di valutazione dei capitolati
    \end{itemize}
    \item[2)] Spostamento della cartella materiali di studio all’interno della cartella documenti.
\end{itemize} 
\subsection{Modifica dei documenti}
La struttura dei documenti era quasi assente quindi si è deciso di modificare i documenti aggiungendo un'intestazione e un sommario; inoltre i seguenti documenti sono stati modificati per rispettare i requisiti richiesti:
\begin{itemize}
    \item "Documento Impegni"
    \item "Lettera di presentazione"
\end{itemize}
Infine è stato scritto il documento "Valutazione del capitolato".
\end{document}
