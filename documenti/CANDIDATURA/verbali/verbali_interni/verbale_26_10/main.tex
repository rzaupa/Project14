\documentclass[12pt, oneside]{article} 
\usepackage{amsmath, amsthm, amssymb, calrsfs, wasysym, verbatim, bbm, color, graphicx, geometry, fancyhdr}
\usepackage[italian]{babel}

\geometry{tmargin=.75in, bmargin=.75in, lmargin=.75in, rmargin = .75in}


\author{RAMtastic6}

%Intestazione
\pagestyle{fancy}
\fancyhf{}
\fancyhead[R]{Gruppo 14 RAMtastic6\\ramtastic6@gmail.com}
\fancyfoot[C]{\thepage}

% Intestazione documento
\begin{document}
% Salta la prima pagina per l'intestazione
\thispagestyle{empty}
\title{Riunione di team}
\date{26 ottobre 2023} % cambiare la data del verbale in questo slot

\maketitle
\begin{figure}[h]
	\centering
	\includegraphics[scale=0.3]{logo.png}
	\label{}
\end{figure}
\begin{center}
    email: ramtastic6@gmail.com
\end{center}

% Informazioni sul documento
\section*{Informazioni sul documento}
\begin{tabular}{ll}
Versione: & 1.0 \\
Redattori: & Filippo T. \\
Verificatori: & Michele Z. \\ 
\end{tabular}
\newpage

% Sommario
\tableofcontents
\newpage

% Contenuto
\section{Partecipanti}

\begin{center}
	\begin{tabular}{ | c | c | c | }
		\hline
		NOME & DURATA \\ 
		\hline
		Leonardo Basso & 1h \\  
		\hline 
		Davide Brotto & 1h \\   
        \hline
        Riccardo Zaupa & 1h \\   
		\hline
        Michele Zambon & 1h \\   
        \hline
        Samuele Visentin & 1h \\
        \hline
        Filippo Tonietto & 1h \\
        \hline
	\end{tabular}
\end{center}

\section{Riassunto dell'incontro}
Durante l'incontro tutti i partecipanti si sono presentati e hanno esposto le loro impressioni sui capitolati esposti. Dopo essersi confrontati, i partecipanti hanno individuato come prima scelta il capitolato 3, \textit{Easy Meal}, proposto dall'azienda IMOLA INFORMATICA. Come seconda e terza opzione, invece, sono stati selezionati i capitolati 5 e 9. \\
In seguito, sono stati individuati alcuni dubbi riguardo sia il progetto in sé, che sulla mole di documenti da fornire per la candidatura. Riguardo il progetto, il principale dubbio emerso è stato sulle tecnologie da utilizzare, in particolare sulla parte del "sito web responsivo"; si è anche pensato di chiedere direttamente all'azienda consigli su come organizzarsi internamente. \\
In ultimo, è stato creato e impostato il profilo GitHub del gruppo, ed è stata fatta una veloce overview sul funzionamento di Gitflow da parte dei membri del team che già ne avevano dimestichezza.
\end{document}
